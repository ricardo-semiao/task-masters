% !TEX root = ps5_text.tex

\documentclass[12pt]{article}

% Geometry
\usepackage[a4paper, left=3cm, right=2.5cm, top=2.5cm, bottom=3cm]{geometry}

% Font encoding
\usepackage[utf8]{inputenc} % UTF-8 encoding
\usepackage[T1]{fontenc} % Font encoding
\usepackage{times}

% Math packages
\usepackage{amsmath} % Basic math symbols and environments
\usepackage{amssymb} % Additional math symbols
\usepackage{amsfonts} % Math fonts

% Text packages
\usepackage{parskip}
\setlength{\parskip}{1em}
\usepackage{hyperref}
\hypersetup{
    colorlinks=true,
    linkcolor=blue,
}

% Pictures
\usepackage{graphicx}
\usepackage{float}

% Lists
\usepackage{enumitem}
\setlist[itemize]{itemsep = -0.5em, topsep = -0.5em}

% Bibliography
%\usepackage{cite}

% Loops:
\usepackage{pgffor}

% Extra commands:
\makeatletter
\renewcommand{\maketitle}{
  \begin{center}
    {\Huge \@title}\\[2em]
    {\large \@author \hfill \@date}\\[2em]
  \end{center}
}
\makeatother


% Title and author
\title{Econometrics II - Problem Set 5}
\author{Ricardo Semião e Castro}
\date{06/2024}


\begin{document}

\maketitle

\section*{Question 1}

\subsection*{Item 1.}

The gradient is:

$$
E[\left(-\frac{1}{\lambda^2} ~~~~ -2\frac{X}{\lambda^2}\right)']
$$
$$
\left(-\frac{1}{\lambda^2} ~~~~ -2\frac{E[X]}{\lambda^2}\right)'
$$


\subsection*{Item 2.}
The results can be seen below:


\begin{table}[H] \centering 
  \caption{} 
  \label{tb:gmm_exp} 
\begin{tabular}{@{\extracolsep{5pt}} cccccc} 
\\[-1.8ex]\hline 
\hline \\[-1.8ex] 
 & term & estimate & std.error & statistic & p.value \\ 
\hline \\[-1.8ex] 
1 & Lambda & 4.970 & Inf & 0 & 1 \\ 
\hline \\[-1.8ex] 
\end{tabular} 
\end{table} 




\section*{Question 2}

\subsection*{Item 1.}
We can see that the exogeneity assumption does not hold, $E[Y_{t-1}U_t] \neq 0$:

$$
E[Y_{t-1}U_t] = E[(\phi_1 Y_{t-2} + U_{t-1} )U_t] = E[(\phi^2_1 Y_{t-3} + \phi_1U_{t-2} + U_{t-1})U_t]
$$
$$
E[Y_{t-1}U_t] = \theta_2(\phi_1 + \theta_1) + \theta_1 \neq 0
$$

\subsection*{Item 2.}
We can see that the exogeneity assumption does not hold, $E[Y_{t-2}U_t] \neq 0$:

$$
E[Y_{t-2}U_t] = E[(\phi_1 Y_{t-3} + \epsilon_{t-2} + \theta_1\epsilon_{t-3} + \theta_2\epsilon_{t-4})(\theta_1\epsilon_{t-1} + \theta_2\epsilon_{t-2})] = \theta_2 \neq 0
$$


\subsection*{Item 4.}
The results can be seen below:


\begin{table}[H] \centering 
  \caption{} 
  \label{tb:gmm_arma} 
\begin{tabular}{@{\extracolsep{5pt}} cccccc} 
\\[-1.8ex]\hline 
\hline \\[-1.8ex] 
 & term & estimate & std.error & statistic & p.value \\ 
\hline \\[-1.8ex] 
1 & (Intercept) & 0.00521 & 0.00359 & 1.450 & 0.146 \\ 
2 & Lag 1       & 0.19254 & 0.02034 & 9.464 & 2.950e-21 \\ 
\hline \\[-1.8ex] 
\end{tabular} 
\end{table} 





\section*{Question 3}

\subsection*{Item 1., 2., and 3.}
In the code.


\subsection*{Item 4.}
The results can be seen below:


\begin{table}[H] \centering 
  \caption{} 
  \label{tb:gmm_capm} 
\begin{tabular}{@{\extracolsep{5pt}} cccccc} 
\\[-1.8ex]\hline 
\hline \\[-1.8ex] 
 & term & estimate & std.error & statistic & p.value \\ 
\hline \\[-1.8ex] 
1 & ABEV3 (Intercept) & -0.001634 & 0.00141 & -1.1522 & 0.249 \\ 
2 & BBDC3 (Intercept) &  0.001402 & 0.00120 &  1.1646 & 0.244 \\ 
3 & ITUB3 (Intercept) &  0.000173 & 0.00134 &  0.1285 & 0.897 \\ 
4 & ABEV3 BVSP & 0.851 & 0.0796 & 10.691 & 1.118e-26 \\ 
5 & BBDC3 BVSP & 1.130 & 0.0598 & 18.885 & 1.488e-79 \\ 
6 & ITUB3 BVSP & 1.066 & 0.1078 & 9.8919 & 4.509e-23 \\ 
\hline \\[-1.8ex] 
\end{tabular} 
\end{table} 



\subsection*{Item 5.}
The joint hypothesis is every intecept having a null value of $0$ de DFs are $(244, 3)$. The results are as below:


\begin{table}[H] \centering 
  \caption{} 
  \label{tb:test_capms} 
\begin{tabular}{@{\extracolsep{5pt}} ccccccccc} 
\\[-1.8ex]\hline 
\hline \\[-1.8ex] 
 & term & null.value & estimate & std.error & statistic & p.value\\ 
\hline \\[-1.8ex] 
1 & ABEV3 (Intercept) & 0 & -0.00163 & 0.00141 & 2.983 & 0.394\\ 
2 & BBDC3 (Intercept) & 0 &  0.00140 & 0.00120 & 2.983 & 0.394\\ 
3 & ITUB3 (Intercept) & 0 &  0.00017 & 0.00134 & 2.983 & 0.394\\ 
\hline \\[-1.8ex] 
\end{tabular} 
\end{table} 


With the p-value of $0.394$, we do not reject the null.

\end{document}
